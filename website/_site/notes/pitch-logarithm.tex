\documentclass[10pt]{article}
\setlength\parindent{0pt}
\setlength\parskip{1em}
\usepackage{fullpage}
\usepackage{epsf}
\usepackage{amsmath}
\usepackage{graphicx}
\usepackage{fullpage}
\usepackage{graphicx}
\usepackage{musixtex}
\begin{document}

\begin{center} \Large Pitch-Frequency Tutorial \end{center}

The purpose of this guide is to figure out the connection between the pitch and the fundamental frequency of a note. The precise meanings of these terms are:

\begin{itemize}
  \item {\bf Pitch:} How high a note sounds to our ears. This is a musical term; for instance, we would say that the note $A_4$ has a pitch one octave higher than $A_3$.
  \item {\bf Fundamental frequency:} The lowest frequency of a musical tone, which is also the spacing between the harmonics produced. For instance, we might say that a
    note has a fundamental frequency of 440 Hz. This is a scientific term.
\end{itemize}

As we will see, it's only {\it relative} frequency that matters for things related to harmony. So we are free to choose a starting point: we need to pick the 
frequency of {\it one} note on the piano keyboard. The mathematics of harmony will let us work out all the others (in an exercise you will complete today!).

{\bf Discuss with your neighbors how we should do this, then we'll talk about it. On the jumbo-sized grand staff provided, fill in the frequency of our starting point.}


\vspace{1in}

\section{Figuring out the frequency difference of an octave}

We have lots of musical instruments in this room: a guitar (which some of you will know how to play, but anyone can poke at), a piano (which you all can play!), and those of you with 
laptops or tablets can Google ``online piano''. 

{\bf Using one of these instruments -- or your own voices! -- figure out how the fundamental frequency of a note changes when you go up by an octave.}

Your statement will take one of two forms:

\begin{itemize}
  \item \it When the pitch of a note increases by an octave, you add \underline{\hspace{1in}} Hz to the frequency.
  \item \it When the pitch of a note increases by an octave, you multiply the frequency by a factor of \underline{\hspace{1in}}.
\end{itemize}

If you already know the answer to this, don't tell your group; the point is to figure it out. But you may help them figure it out!

{\bf Once you've figured this out, fill in the fundamental frequency of all the A's on your staff, based on the frequency of middle A you chose earlier.}

\medskip


\newpage

\section{Math language}

{\bf What is the ratio between the frequency of the note $C_6$ (the highest note women are usually asked to sing) and the frequency of the note $C_2$ (the lowest
note men are usually asked to sing?}

\vspace{1.5in}

{\bf In algebra terms, how would you write the mathematical relationship between the number of octaves $\mathcal O$ and the frequency ratio $R$? That is,
$R$ is some function of $\mathcal O$; what is it? (Hint: how do you write ``multiply by two this many times?'')}

\vspace{1.5in}

{\bf Now invert the previous formula, to get $\mathcal O$ as a function of $R$. This involves logarithms.} If you 
already know how to do this, go ahead and do it. If you don't, ask me and I'll show you what a logarithm is. It's far simpler than your algebra teacher
made it out to be. :)


\newpage

\section{Figuring out the frequency difference of a semitone/halfstep}

Now, we need to fill in the other notes. Remember that a halfstep is 1/12 of an octave, a fifth is 7/12 of an octave, and so on.
In symbols, we could say that $N$ halfsteps is equal to $N/12$ octaves -- or $\mathcal O = N/12$.

Recall that on the previous page you found out that, if note 2 is $O$ octaves above note 1, you know that 

$$
\large \frac{\rm Frequency\, of\, note\, 2}{\rm Frequency\,of\,note\,1} = 2^\mathcal O
$$

This means that

$$
\large \frac{f_2}{f_1} = 2^{N/12}
$$

This formula lets you calculate the frequency of any pitch on the piano. As a warmup, calculate the frequency of middle C, and fill it in on your staff.
(Remember, you know the frequency of middle A is 440 Hz. So let that be your $f_1$, and solve for $f_2$.)

\vspace{1in}

\bf So, if the interval of an octave represents a factor of 2 in frequency, what factor does the interval of a halfstep represent? Make a statement 
similar to your previous one:

\it When the pitch of a note increases by a semitone, you multiply the frequency by a factor of \underline{\hspace{1in}}.
\rm
\vspace{1in}

Now, get into groups of four and work in pairs.

One pair of people should calculate frequencies from middle A ($A_4$, which is 440 Hz) all the way up to $C_6$ (that is, two octaves
above middle C), and record those on the grand staff. (You only need one copy of this.) Then, calculate the frequency of $C_2$ (two octaves
below middle C) and record it.

The other pair of people should calculate frequencies from middle A down to $C_3$ (that is, one octave below middle C) and record those.

One group of four will work on the board, so we have a record for the whole class to see.

\newpage

\section{The chord of nature}

We noticed earlier that musical tones consist of lots of frequencies that are all multiples of the fundamental.

Take a low note: $C_2$, the lowest note a cello can play, and the lowest note men are usually asked to sing. 

\begin{itemize}
  \item What is its fundamental frequency?

  \item What are the other frequencies produced when someone plays/sings this note?

  \item Using our labeled staff on the board, what {\it other notes} have these fundamental frequencies?

  \item Two people who have never played the piano before: come find and play those notes all at once on the keyboard!
\end{itemize}
\end{document}
