\documentclass[10pt]{article}
\setlength\parindent{0pt}
\usepackage{fullpage}
\usepackage{amsmath}
\usepackage{epsf}
\usepackage{hyperref}
\usepackage{musixtex}
\usepackage{graphicx}
\setlength{\parskip}{4mm}
\def\LL{\left\langle}   % left angle bracket
\def\RR{\right\rangle}  % right angle bracket
\def\LP{\left(}         % left parenthesis
\def\RP{\right)}        % right parenthesis
\def\LB{\left\{}        % left curly bracket
\def\RB{\right\}}       % right curly bracket
\def\PAR#1#2{ {{\partial #1}\over{\partial #2}} }
\def\PARTWO#1#2{ {{\partial^2 #1}\over{\partial #2}^2} }
\def\PARTWOMIX#1#2#3{ {{\partial^2 #1}\over{\partial #2 \partial #3}} }
\newcommand{\BE}{\begin{displaymath}}
\newcommand{\EE}{\end{displaymath}}
\newcommand{\BNE}{\begin{equation}}
\newcommand{\ENE}{\end{equation}}
\newcommand{\BEA}{\begin{eqnarray}}
\newcommand{\EEA}{\nonumber\end{eqnarray}}
\newcommand{\EL}{\nonumber\\}
\newcommand{\la}[1]{\label{#1}}
\newcommand{\ie}{{\em i.e.\ }}
\newcommand{\eg}{{\em e.\,g.\ }}
\newcommand{\cf}{cf.\ }
\newcommand{\etc}{etc.\ }
\newcommand{\Tr}{{\rm tr}}
\newcommand{\etal}{{\it et al.}}
\newcommand{\OL}[1]{\overline{#1}\ } % overline
\newcommand{\OLL}[1]{\overline{\overline{#1}}\ } % double overline
\newcommand{\OON}{\frac{1}{N}} % "one over N"
\newcommand{\OOX}[1]{\frac{1}{#1}} % "one over X"
\newcommand{\vsi}{\vspace{1in}}
\newcommand{\vshi}{\vspace{0.5in}}

\begin{document}
\pagenumbering{gobble}
\begin{center} \sc \Large Quiz -- Jan 29\end{center}

  \begin{enumerate}
    \item Suppose that you are playing music written in the early 1700's, where the fundamental frequency of middle A (written $A_5$) is 420 Hz instead of 440 Hz as it is in modern
      music.\footnote{The most common tuning for Baroque music is $A_5 = 415$ Hz, but this makes the math harder for you all to do in your heads. :)}  What are the fundamental frequencies of the notes:

      \begin{enumerate}
	\item $A_3$ (two octaves below)
	  \vshi
	\item $A_4$ (one octave below)
	  \vshi

	\item $A_6$ (one octave above)
	  \vshi

	\item $A_7$ (two octaves above)
	  \vshi

	\item $E_6$ (a fifth, seven half-steps, above)
	  \vshi

      \end{enumerate}

    \item We say that musical intervals are ``equally-spaced'', in the sense that all half-steps (or octaves, or fifths, or any interval) on the piano keyboard
      are the same size. However, in looking at the answers to the above, this might not seem to be the case. For instance, the difference in Hz between the fundamentals of 
      $A_3$ and $A_4$ is much smaller than the difference in Hz between $A_6$ and $A_7$. 

      How can you reconcile these two statements?

  \end{enumerate}
\end{document}
