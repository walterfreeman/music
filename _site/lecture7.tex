% DPF 09 talk on strangeness in nucleon

\documentclass[10pt]{beamer}
\usepackage{amsmath}
\usepackage{mathtools}
\usefonttheme{professionalfonts} % using non standard fonts for beamer
\usefonttheme{serif} % default family is serif

%\documentclass[12pt]{beamerthemeSam.sty}
\usepackage{epsf}
%\usepackage{pstricks}
%\usepackage[orientation=portrait,size=A4]{beamerposter}
\geometry{paperwidth=160mm,paperheight=120mm}
%DT favorite definitions
\def\LL{\left\langle}	% left angle bracket
\def\RR{\right\rangle}	% right angle bracket
\def\LP{\left(}		% left parenthesis
\def\RP{\right)}	% right parenthesis
\def\LB{\left\{}	% left curly bracket
\def\RB{\right\}}	% right curly bracket
\def\PAR#1#2{ {{\partial #1}\over{\partial #2}} }
\def\PARTWO#1#2{ {{\partial^2 #1}\over{\partial #2}^2} }
\def\PARTWOMIX#1#2#3{ {{\partial^2 #1}\over{\partial #2 \partial #3}} }

\def\rightpartial{{\overrightarrow\partial}}
\def\leftpartial{{\overleftarrow\partial}}
\def\diffpartial{\buildrel\leftrightarrow\over\partial}

\def\BI{\begin{itemize}}
\def\EI{\end{itemize}}
\def\BE{\begin{displaymath}}
\def\EE{\end{displaymath}}
\def\BEA{\begin{eqnarray*}}
\def\EEA{\end{eqnarray*}}
\def\BNEA{\begin{eqnarray}}
\def\ENEA{\end{eqnarray}}
\def\EL{\nonumber\\}


\newcommand{\map}[1]{\frame{\frametitle{\textbf{Course map}}
\centerline{\includegraphics[height=0.86\paperheight]{../../map/#1.png}}}}
\newcommand{\wmap}[1]{\frame{\frametitle{\textbf{Course map}}
\centerline{\includegraphics[width=0.96\paperwidth]{../../map/#1.png}}}}

\newcommand{\etal}{{\it et al.}}
\newcommand{\gbeta}{6/g^2}
\newcommand{\la}[1]{\label{#1}}
\newcommand{\ie}{{\em i.e.\ }}
\newcommand{\eg}{{\em e.\,g.\ }}
\newcommand{\cf}{cf.\ }
\newcommand{\etc}{etc.\ }
\newcommand{\atantwo}{{\rm atan2}}
\newcommand{\Tr}{{\rm Tr}}
\newcommand{\dt}{\Delta t}
\newcommand{\op}{{\cal O}}
\newcommand{\msbar}{{\overline{\rm MS}}}
\def\chpt{\raise0.4ex\hbox{$\chi$}PT}
\def\schpt{S\raise0.4ex\hbox{$\chi$}PT}
\def\MeV{{\rm Me\!V}}
\def\GeV{{\rm Ge\!V}}

%AB: my color definitions
%\definecolor{mygarnet}{rgb}{0.445,0.184,0.215}
%\definecolor{mygold}{rgb}{0.848,0.848,0.098}
%\definecolor{myg2g}{rgb}{0.647,0.316,0.157}
\definecolor{abtitlecolor}{rgb}{0.0,0.255,0.494}
\definecolor{absecondarycolor}{rgb}{0.0,0.416,0.804}
\definecolor{abprimarycolor}{rgb}{1.0,0.686,0.0}
\definecolor{Red}           {cmyk}{0,1,1,0}
\definecolor{Grey}           {cmyk}{.7,.7,.7,0}
\definecolor{Blue}          {cmyk}{1,1,0,0}
\definecolor{Green}         {cmyk}{1,0,1,0}
\definecolor{Brown}         {cmyk}{0,0.81,1,0.60}
\definecolor{Black}         {cmyk}{0,0,0,1}

\usetheme{Madrid}


%AB: redefinition of beamer colors
%\setbeamercolor{palette tertiary}{fg=white,bg=mygarnet}
%\setbeamercolor{palette secondary}{fg=white,bg=myg2g}
%\setbeamercolor{palette primary}{fg=black,bg=mygold}
\setbeamercolor{title}{fg=abtitlecolor}
\setbeamercolor{frametitle}{fg=abtitlecolor}
\setbeamercolor{palette tertiary}{fg=white,bg=abtitlecolor}
\setbeamercolor{palette secondary}{fg=white,bg=absecondarycolor}
\setbeamercolor{palette primary}{fg=black,bg=abprimarycolor}
\setbeamercolor{structure}{fg=abtitlecolor}

\setbeamerfont{section in toc}{series=\bfseries}

%AB: remove navigation icons
\beamertemplatenavigationsymbolsempty
\title[Newton's Law of Motion]{
  \textbf {Newton's Law of Motion}\\
%\centerline{}
%\centering
%\vspace{-0.0in}
%\includegraphics[width=0.3\textwidth]{propvalues_0093.pdf}
%\vspace{-0.3in}\\
%\label{intrograph}
}

\author[W. Freeman] {Physics 211\\Syracuse University, Physics 211 Spring 2018\\Walter Freeman}

\date{\today}

\begin{document}

\frame{\titlepage}

\frame{\frametitle{\textbf{Announcements}}
\BI
\Large
\item Clinic hours today: 1:30-5
\item Prof. Plourde is out of town and will return tomorrow
\item HW3 is posted
\medskip
\pause
\item An opportunity for you to get some minor extra credit and help uphold academic honesty...
\pause
\medskip
\BI
\item Two people sitting next to each other turned in suspiciously similar papers
\item We are pretty sure that this isn't a coincidence, but want to make sure
\item We can study this rigorously based on how often people made particular choices in solving a problem
\item Give your exams to your TA tomorrow; we'll look at them, give you +1 point, and give them back
\EI
\EI
}

\frame{\frametitle{\textbf{Exam 1}}
\Large
\begin{center}
Any questions?
\end{center}
}

\frame{\frametitle{\textbf{Newton's laws}}
    \Large

    \centerline{Newton's second law: $\vec F = m\vec a$}

\bigskip

    \large
    \BI
  \item{Forces on an object cause it to accelerate}
  \item{The larger the force, the larger the acceleration}
  \item{The larger the mass, the smaller the acceleration}
  \item{You intuitively know this already}
    \pause
  \item{No forces $\rightarrow$ no acceleration: {\color{Red}not necessarily no motion!}}

    \bigskip
    \bigskip
    \bigskip

\centerline{\Large Newton's third law: Forces come in pairs}

\bigskip

\item{``If A pushes on B, B pushes back on A''}
\item{Very important to be clear about what forces you're talking about}
    \EI
   }


\frame{
\Large
Which of the following is/are {\it not} an example of Newton's third law?

\BI
\item A: a subway car accelerates forward; you are thrown back
\item B: the propeller on an airplane pushes the air backwards; the air pushes the airplane forwards
\item C: an elevator accelerates upward; passengers are pushed downward
\item D: the Earth's gravity pulls downward on me; my gravity pulls upward on the Earth
\item E: a rocket pushes downward on its exhaust; the exhaust pushes upward on the rocket
\EI
}




   \frame{\frametitle{\textbf{Newtons}}
     \Large{\centerline{We need a new unit for force: the newton}}

     \bigskip
     \bigskip

     \centerline{     $\vec F = m \vec a \rightarrow$ Force has dimensions kg $\rm m/\rm s^2$}
\large
     \bigskip
     \bigskip
\BI
\item{1 N = 1 kg $\rm m/\rm s^2$: about the weight of an apple}

\item{4 N is about a pound}
     \item{9.8 N is the weight of a kilogram}
       \EI
   }

   \frame{\frametitle{\textbf{Force is a vector}}
     \Large
     \centerline{$\vec F = m\vec a$}
     \large
     \BI
   \item{Force is a {\it vector}}
   \item{Multiple forces on an object add like vectors do}
   \item{Really, we should write}

     \Large
     \centerline{$\sum \vec F = m\vec a$}

     \pause
\small

\bigskip
\bigskip
\bigskip
\bigskip
\bigskip
     \centerline{     (dragging disc demo)}
\EI
   }

   \frame{\frametitle{\textbf{What is a force?}}
    A force is anything that pushes or pulls something:
    \BI
  \item{Gravity: $F = mg$, so $mg = ma \rightarrow a = g$}
    \BI
  \item{Gravity pulls down on everything (on Earth) with a force $mg$, called its weight}
  \item{If something isn't accelerating downward, some other force must balance its weight}
    \EI
    \EI
  }

  \frame{\frametitle{\textbf{What is a force?}}
    A force is anything that pushes or pulls something:
    \BI
  \item{Gravity: $F = mg$, so $mg = ma \rightarrow a = g$}
  \item{``Normal force'': stops things from moving through each other}
    \BI
  \item{Are there normal forces on me right now?}
 \pause
  \item{However big it needs to be to stop objects from sliding through each other}
  \item{Directed ``normal'' (perpendicular) to the surface}
  \item{Really caused by electric force/Pauli exclusion principle}
    \EI
    \EI
  }


  \frame{\frametitle{\textbf{What is a force?}}
    A force is anything that pushes or pulls something:
    \BI
  \item{Gravity: $F = mg$, so $mg = ma \rightarrow a = g$}
  \item{``Normal force'': stops things from moving through each other}
  \item{Tension: ropes pull on both sides equally}
    \BI
  \item{What are the forces in a contest of tug-of-war?}
    \pause
  \item{What about the forces on the people?}
    \EI
  \item{Friction: a force opposes things sliding against each other}
    \pause
  \item{Electromagnetic forces, nuclear forces, radiation pressure...}
    \pause
  \item{\color{Red}Acceleration is not a force!}
  \item{\color{Red}... it's the {\it result} of forces}
    \EI
  }

\frame{\frametitle{\textbf{One particular force: gravity}}

\Large

Gravity exerts a downward force on all objects (on Earth), with a magnitude of $mg$.

\bigskip

In symbols: $\vec F_g = mg$ downward.

\bigskip
\pause

Why is the acceleration of a falling object $g$ downward?
 
\BI
\item A: Because $g$ is the acceleration of all objects within Earth's gravitational field
\item B: Solve Newton's law: $\vec F = m \vec a \rightarrow mg (-\hat j) = m\vec a \rightarrow \vec a = -g\hat j$
\item C: Because the definition of $g$ is the acceleration that a falling object undergoes
\item D: It's only $g$ if there are no other forces besides gravity acting on it\EI
} 

  \frame{\frametitle{\textbf{Force diagrams}}
  \BI
  \item{Lots of forces, easy to get confused}
  \item{Draw a picture!}
    \centerline{\includegraphics[width=0.6\textwidth]{cruise.png}}
    \pause
  \item{Each object feeling forces gets a separate diagram}
  \item{Label each force and its direction}
  \item{These are also called ``free body diagrams''}
    \EI
  \bigskip
  \bigskip
  \bigskip
  \bigskip
  \pause
  \small
  \centerline{(Examples on document camera)}

  }


\frame{
\Large

Suppose an object is moving in a straight line at a constant speed. Which number of forces could {\it not} be 
acting on it?

\BI
\item A: Zero
\item B: One
\item C: Two
\item D: Three
\item E: Four
\EI

\pause

Suppose an object is moving in a circle at a constant speed. Which number of forces could {\it not} be 
acting on it? (Hint: what is the definition of velocity? Of acceleration?)

\BI
\item A: Zero
\item B: One
\item C: Two
\item D: Three
\item E: Four
\EI
}


  \frame{\frametitle{\textbf{Sample questions}}
    \BI
    \Large
  \item{What forces act on a car?}
    \pause
  \item{Which forces are bigger or smaller if it's driving at a constant speed?}
    \pause
  \item{Which forces are bigger or smaller if it's slowing down?}
    \pause
  \item{A 1000 kg car slows from 20 m/s to a stop over 5 sec. If we ignore air resistance, what force is required to do this?}


    \pause

\bigskip
\bigskip
\bigskip
\normalsize
\centerline{(Use $\vec F = m \vec a$ to connect force to acceleration, and then kinematics to connect acceleration to motion)}

    \EI
  }

\frame{\frametitle{\textbf{An important note}}
\large
\BI
\item{Only {\it real physical things} are forces}
\pause
\item{Acceleration is not a force}
\item{``Net force'' is not a force (it's the sum of them)}
\item{Velocity certainly isn't a force}
\pause
\item{If two things don't touch, or interact by gravity, electricity, etc., they don't
exchange forces}
\pause
\item{``A force is something that can send you to the doctor''}
\EI
}

\frame{\frametitle{\textbf{A sample problem}}
\Large
A stack of two books sits on a table. Each book weighs 10 newtons. Draw a force diagram
for each one, and calculate the size of all the forces.

\bigskip
\bigskip
\bigskip

(Your answer should match what you know about how this works!)
}

  \frame{\frametitle{\textbf{Summary}}
    \Large
    \BI
  \item{Forces: anything that pushes or pulls}
  \item{Forces cause accelerations: $\sum \vec F = m \vec a$}
    \BI
  \item{If $\sum \vec F = 0$, $\vec a = 0$: motion at a constant velocity}
  \EI
  \item{Forces come in pairs: if A pushes on B, B pushes back on A}
  \item{It's the vector sum $\sum \vec F$ that matters}
  \item{Draw force diagrams to keep all of this straight}
    \EI
  }



  \end{document}

