\documentclass[12pt]{article}
\setlength\parindent{0pt}
\usepackage{fullpage}
\usepackage{hyperref}
\setlength{\parskip}{4mm}
\def\LL{\left\langle}   % left angle bracket
\def\RR{\right\rangle}  % right angle bracket
\def\LP{\left(}         % left parenthesis
\def\RP{\right)}        % right parenthesis
\def\LB{\left\{}        % left curly bracket
\def\RB{\right\}}       % right curly bracket
\def\PAR#1#2{ {{\partial #1}\over{\partial #2}} }
\def\PARTWO#1#2{ {{\partial^2 #1}\over{\partial #2}^2} }
\def\PARTWOMIX#1#2#3{ {{\partial^2 #1}\over{\partial #2 \partial #3}} }
\newcommand{\BI}{\begin{itemize}}
\newcommand{\EI}{\end{itemize}}
\newcommand{\BE}{\begin{displaymath}}
\newcommand{\EE}{\end{displaymath}}
\newcommand{\BNE}{\begin{equation}}
\newcommand{\ENE}{\end{equation}}
\newcommand{\BEA}{\begin{eqnarray}}
\newcommand{\EEA}{\nonumber\end{eqnarray}}
\newcommand{\EL}{\nonumber\\}
\newcommand{\la}[1]{\label{#1}}
\newcommand{\ie}{{\em i.e.\ }}
\newcommand{\eg}{{\em e.\,g.\ }}
\newcommand{\cf}{cf.\ }
\newcommand{\etc}{etc.\ }
\newcommand{\Tr}{{\rm tr}}
\newcommand{\etal}{{\it et al.}}
\newcommand{\OL}[1]{\overline{#1}\ } % overline
\newcommand{\OLL}[1]{\overline{\overline{#1}}\ } % double overline
\newcommand{\OON}{\frac{1}{N}} % "one over N"
\newcommand{\OOX}[1]{\frac{1}{#1}} % "one over X"

\begin{document}
\Large
\centerline{\sc{Physics of Music Homework 2}}
\centerline{\large Due Monday, 12 February, printed and handed to me at the beginning of class}
\normalsize

\bigskip
\bigskip

   In class, we observed that the frequency ratio corresponding to the interval of an octave is 2, and the frequency ratio corresponding to $\mathcal O$ octaves is $2^\mathcal O$.
    Observing that a halfstep is 1/12 of an octave, we note that 

    $$\frac{\rm fundamental\,of\,note\,\beta}{\rm fundamental\,of\,note\,\alpha} = \frac{f_\beta}{f_\alpha} = 2^{N/12}$$

    where $N$ is the interval in halfsteps from $\alpha$ up to $\beta$. (Note that if $\alpha$ is higher than $\beta$, the exponent is negative, but this doesn't break anything,
    since a negative exponent just gives a fraction less than one.)


   In class, we had students calculate the pitches of the notes whose fundamentals most closely matched the overtones of the note $C_2$, and saw that the harmonics of $C_2$
    spell out a chord with $C_2$, $C_3$, $G_3$, $C_4$, $E_4$, $G_4$, $B\flat_4$, and $C_5$. This is sometimes called the ``chord of nature''.

  

    \begin{enumerate}
      \item What are the named musical intervals between these notes? (For instance, what is the distance from $C_2$ to $C_3$, from $C_3$ to $G_3$, and so forth?)

	A reference to the names of intervals:

	\BI
      \item One halfstep: ``halfstep'', ``semitone'', ``minor second''
      \item Two halfsteps: ``whole step'', ``major second''
      \item Three halfsteps: ``minor third''
      \item Four halfsteps: ``major third''
      \item Five halfsteps: ``fourth'' or ``perfect fourth''
      \item Six halfsteps: ``tritone''
      \item Seven halfsteps: ``fifth'' or ``perfect fifth''
      \item Eight halfsteps: ``minor sixth''
      \item Nine halfsteps: ``major sixth''
      \item Ten halfsteps: ``minor seventh''
      \item Eleven halfsteps: ``major seventh''
      \item Twelve halfsteps: ``octave''
	\EI

	Start making a table of these -- ultimately, you'll be creating something like this:

	    \small
	  \begin{tabular}{| c | c | c | c | c |}
	    \hline
	    Interval            & Name of interval & Frequency ratio & Frequency ratio                     & Ratio of these                               \\
	                        &                  &(equitempered tuning) &                 (harmonic sequence) &            \\
	    \hline
	    $C_2 \rightarrow C_3$ &      &                            &                                     &                                  \\
	    \hline
	    $C_3 \rightarrow G_3$ &      &                            &                                     &                                  \\
	    \hline
	    $G_3 \rightarrow C_4$ &      &                            &                                     &                                  \\
	    \hline
	    $C_4 \rightarrow E_4$ &      &                            &                                     &                                  \\
	    \hline
	    $E_4 \rightarrow G_4$ &      &                            &                                     &                                  \\
	    \hline4
	    $G_4 \rightarrow B^\flat_4$ &  &                                &                                     &                                  \\
	    \hline
	    $B\flat_4 \rightarrow C_5$ &  &                                &                                     &                                  \\
	    \hline
	  \end{tabular}


     \item Calculate, using the formula from above, what the frequency ratio is that corresponds to each of those intervals in the ``chord of nature'', based on our previous
       ideas about how to tune the piano. (This is called ``equitempered tuning''.)
       For example, the interval from $C_3$ to $G_3$ is a perfect fifth, and has the ratio $2^{7/12} = 1.4983$. (Carry these calculations to four decimal places. This will matter later!)

	Add this to your table.

     \item We have a choice here, however! In class, we decided that we would tune our piano by slicing the octave into twelve equal parts. 
       (This is the tuning system used by modern pianos.) However, perhaps we should instead define
       these intervals by the frequencies of the overtones of $C_2$, giving us intervals that correspond to fractions of small integers. (For instance, a fifth is the distance 
	from the third harmonic to the second harmonic = $3/2 = 1.5000$.

	Do this for all of the intervals that appear in the ``chord of nature'', and add these to your table.

      \item Now, for each interval, you will have calculated its size in two different ways: once where we allow the harmonic sequence to be the ``authority'' on how big intervals should be,
	and once where we allow our earlier procedure -- ``cut the octave into twelve equal pieces and thus define a semitone as $2^{1/12}$'' -- to be the authority.

	As a way of computing how much they agree or disagree, compute the ratio between the two methods. (For instance, the degree of disagreement in our definition
	of a fifth is 1.5000 / 1.4983 = 1.0011.) Add this to your table.

	Which intervals show more disagreement? Which ones show less?

      \item {\it Extra credit, related to the last question:} you will notice that one interval, in particular, is very far off: the one from $C$ to $B\flat$, which 
	forms the seventh of the C7 chord we used as our example.

	It is common practice in blues music to lower certain notes, called ``blue notes'', by an amount smaller than a semitone when the performer is playing an instrument
	without fixed pitch. For instance, in the key of C, the notes $E\flat$ and $B\flat$ (and sometimes others) may be lowered a small amount. Why do you think blues players
	do this? Is this an ``unnatural'' tuning, or is it more ``natural''? (Note: There is some good information about this to be had on Wikipedia. But it is not enough to 
	just tell me what Wikipedia says; relate your answer to the answer to the previous question!)
	
      \item We need to resolve this disagreement.

	The note $A_5$ has a frequency of 880 Hz.

	Calculate the frequency of the note $C^\sharp_6$ -- that is, a major third above $A_5$ -- using both methods. 
	
	Then, open up two tabs of \url {http://www.szynalski.com/tone-generator/}. Set the volume on both to around 15 percent. Also set the waveform of each to ``sawtooth wave'' by clicking on the little representation of a sine curve and changing it.

	Play two combinations together:

	\begin{itemize}
	  \item 880 Hz plus the value you calculated for $C^\sharp_6$ using the ``equitempered tuning'' definition of the major third
	  \item 880 Hz plus the value you calculated for $C^\sharp_6$ using the ``harmonic series'' definition of the major third
	\end{itemize}

	Comment on how they sound. Which one sounds better in tune? Which tuning system does this procedure suggest we use?

      \item Compare the frequencies of the 5th harmonic of $A_5$ and the 4th harmonic of $C^\sharp_6$. Do this for both methods of choosing the frequency of $C^\sharp_6$. 
	You may do this either empirically using your spectrogram or mathematically. Does this relate to your answer to the previous question?

      \item A major third is four halfsteps, so three major thirds should give you an octave. 

	So, this means that if you multiply the ratio for a major third by itself three times, you should get 2.

	Do this for both the ``equitempered tuning'' definition of the major third, and the ``harmonic series'' definition. What do you get?

      \item Write a discussion of a paragraph or two about what your calculations tell you about the mathematics of how we tune our musical instruments.
	Is our tuning system a good one? Are there any alternatives that we might use that would work better? What would be required to make pianos ``perfectly in tune,
	always, regardless of what you play''? This discussion will be a major part of the grade for this assignment, so don't skimp on it!
	




    \end{enumerate}

  \end{document}
